% Chapter 3
\chapter{System Design} % Write in your own chapter title
\label{Chapter 3}
\lhead{Chapter 3. \emph{System Design}} % Write in your own chapter title to set the page header
\section{Software Process Model}
To design and develop a system, the importance of software process
model baggers description. Software process model can be called
the methodology of implementing a software system. Software model
can be defined as follows.

A software process model is an abstract representation of a
process that is a structured set of activities required to develop
a software system. It presents a description of a process from
some particular perspective \citep{bandinelli1993software}. The
general steps \citep{bandinelli1993software} of a software process
model are-


\begin{enumerate}
    \item {Specification - what the system should do and its development
constraints}
    \item {Design and implementation- production of the software
system }
    \item {Validation- checking that the software is what the
customer wants}
    \item {Evolution- changing the software in response to
changing demands}
\end{enumerate}

Now let only three of the popular software models called
waterfall, prototyping and spiral which are mostly used can be
considered


\subsection{Waterfall Model}
The waterfall model is a systematic and sequential approach to
software development that begins with customer specification of
requirements and progresses through planning, modeling,
construction and deployment, culminating in on-going support of
the completed software \citep{pressman2005software}. It is the
oldest paradigm for software engineering.

The phases \citep{bandinelli1993software} of Waterfall Model are-


\begin{itemize}
    \item {Requirements analysis and definition}
    \item {System and software design}
    \item {Implementation and unit testing}
    \item {Integration and system testing }
    \item {Operation and maintenance}
\end{itemize}

Real projects rarely follow it as it is difficult to establish all
requirements explicitly, no room for uncertainty.

These phases of this model can be depicted as following figure
\citep{bandinelli1993software}.

\begin{figure}[htbp]
  \centering
 % \scalebox{0.5}{\includegraphics*[0.5in,8.5in][2.5in,10.5in]{./Figures/logo.eps}}
\includegraphics[height=2in]{./Figures/waterfall.eps}
   % \rule{35em}{0.5pt}
  \caption[Waterfall Model]{Waterfall Model}
  \label{fig:Waterfall}
\end{figure}


When the requirements of a problem are reasonably well understood
and project duration is very short then waterfall model is
suitable. The main drawback of the waterfall model is the
difficulty of accommodating change after the process is underway.
One phase has to be complete before moving onto the next phase.
\subsection{Prototyping Model}
Often, a customer defines a set of general objectives for
software, but does not identify detailed input, processing or
output requirements. In other cases, the developer may be unsure
of the efficiency of an algorithm, the adaptability of an
operating system, or the form that human-machine interaction
should take. In these, and many other situations, prototyping
paradigm may offer the best approach \citep{pressman2005software}.
The phases \citep{bandinelli1993software} of this model are-


\begin{itemize}
    \item {Gather requirements}
    \item {Developer and customer define overall objectives; identify
areas needing more investigation - risky requirements}
    \item {Quick design focusing on what will be visible to
user - input and output formats}
    \item {Use existing program fragments, program
generators to throw together working version}
    \item {Prototype evaluated and requirements
refined}
    \item {Process iterated until customer and
developer satisfied}
    \item {Then throw away prototype
and rebuild system to high quality}
    \item {Alternatively can have
evolutionary prototyping - start with well understood
requirements}
\end{itemize}

The prototyping model can be viewed as following figure
\citep{gelbard2002integrating}-


\begin{figure}[htbp]
  \centering
 % \scalebox{0.5}{\includegraphics*[0.5in,8.5in][2.5in,10.5in]{./Figures/logo.eps}}
\includegraphics[height=2in]{./Figures/prototype.eps}
   % \rule{35em}{0.5pt}
  \caption[Prototype Model]{Prototype Model}
  \label{fig:Prototype}
\end{figure}

\subsection{Spiral Model}
The Spiral Model was defined by Barry Boehm in his article A
Spiral Model of Software Development and Enhancement from 1986.
This model was not the first model to discuss iteration, but it
was the first model to explain why the iteration matters. As
originally envisioned, the iterations were typically 6 months to 2
years long. The spiral model (Boehm, 1988) aims at risk reduction
by any means in any phase. The spiral model is often referred to
as a risk-driven model \citep{boehm1998using}.



\begin{itemize}
    \item { The Spiral Development (or Lifecycle) Model is a systems
development method used in information technology. A different
approach born out of the evolution of the Waterfall Model.
Encompasses the previous models as special cases, and can make use
of a combination of models. Risk analysis asks, "What are the
areas of uncertainty, and what is the probability that they will
slow the progress of development?"}
    \item {It combines the features of the prototyping model and the
waterfall model}
    \item {It is favored for large, expensive, and complicated
models}
\end{itemize}


There are four phases in the "Spiral Model" which are: Planning,
Evaluation, Risk Analysis and Engineering. These four phases are
iteratively followed one after other in order to eliminate all the
problems, which were faced in "The Waterfall Model". Iterating the
phases helps in understating the problems associated with a phase
and dealing with those problems when the same phase is repeated
next time, planning and developing strategies to be followed while
iterating through the phases.

\begin{figure}[htbp]
  \centering
 % \scalebox{0.5}{\includegraphics*[0.5in,8.5in][2.5in,10.5in]{./Figures/logo.eps}}
\includegraphics[height=3in]{./Figures/spiral.eps}
   % \rule{35em}{0.5pt}
  \caption[Spiral Model]{Spiral Model}
  \label{fig:Spiral}
\end{figure}

The goal of the spiral model is to be risk driven, so that the
risks in a given cycle are determined during the Analyzing Risks
section. In order to manage these risks, certain additional
project-specific activities may be planned to address the risks,
such as Requirements Prototyping, if the risk analysis indicates
that the software requirements are not clearly understood. These
project specific risks are termed process drivers. For any process
driver, one or more project specific activities need to be
performed to manage the risk.



\subsection{Model Chosen for Our Project}
Here, it is briefly described why a specific process model for our
work is adopted and why other process models are not followed by.
As our purpose is to implement a hall student information system,
the definition and requirements of the problem is specific and
known so waterfall model is the best suite for the work.


%\begin{itemize}
\subsubsection{The Reason for Choosing Waterfall Model\newline}
From the definition of the waterfall is already known that if the
requirements of a problem are reasonably well understood and
project duration is very short then waterfall model is suitable
.As it is clear that every student hall of the university builds
upon based on some specific established rule and regularities
defined by the higher authority of Hall. So it was possible for us
to specify what our system should or would contain at the start of
the development process. It have been also learned that the first
phase of waterfall model is "Requirements analysis and
definition". So our project could meet the first criteria of
waterfall model fully. Although water fall model or liner
sequential model has some drawbacks yet we have chosen it for the
development of our works. The hall is generally experienced enough
for its problems and requirements from its birth so we would be
able to fix up our requirements at a time. First we analyzed
requirements, second designed it, third generated codes and
finally tested it that is the steps of linear sequential model or
water fall model. For future requirement enhancements and
developments are also provided here.

\subsubsection{ The Reason for Not Choosing Prototyping
Model\newline}
 In prototyping model, prototype is evaluated and
requirements are refined and this process iterated until customer
and developer satisfied. But our web development had to use
cutting-edge, diverse technologies and standards, and integrates
numerous varied components, including varied components, including
traditional and non-traditional software, interpreted scripting
languages, HTML files, databases, images, and other web components
and complex user interfaces. Moreover our project is
content-driven (database-driven). Web based systems development
includes creation and management of the content, as well as
appropriate provisions for subsequent content creation,
maintenance, and management after the initial development and
deployment on a continual basis (in some applications as
frequently as every hour or more).\\
So prototyping model could not be able to provide the appropriate
environment and design step to meet all these requirements.
Besides we knew we have to implement a single entity that is a
collection of logically connected web pages. So navigation would
be an important factor. But in prototyping model, it was not
possible as this facility was absent in this model to implement.

\subsubsection{The Reason for Not Choosing Spiral Model\newline}
Spiral model combines the features of the prototyping model and
the waterfall model. As prototyping model is not chosen. So it was
also clear that this model can not be followed too. As this model
was incomplete, risk-driven and estimation of budget and time
harder to judge at the beginning of the project and our target was
to design and implement a website, it was not felt more interest
with this model.


\section{Database Design}
A database management system is a collection of interrelated data
and a set of programs to access those data. The collection of
data, usually referred to as the database, contains information
relevant to an enterprise. Database systems are designed to manage
large bodies of information. These large bodies of information do
not exist in isolation.
 They are part of the operation of some enterprise whose end product may be information from the database or may be some device or service for which the database plays only a supporting role.
  \citep{silberchatz2006sistema} So a major purpose of a database system is to provide users with an abstract view of the data.

Database design mainly involves the design of the database schema.
The entity-relationship data model is a widely used data model for
database design. The model is intended primarily for the
database-design process. It was developed to facilitate database
design by allowing the specification of an enterprise schema. Such
a schema represents the overall logical structure of the database.
The initial specification of user requirements may be based on
interviews with the database users, and on the designer's own
analysis of the enterprise. The description that arises from this
design phase serves as the basis for specifying the conceptual
structure of the database. Here are the major characteristics of
our designed database for the Hall management system.
\subsection{ER Diagram}
This is a higher-level data model that is based on a perception of
a real world that consists of a collection of basic objects,
called entities and of relationships among these objects. Here the
relationship is the association among several entities. More
specifically, entity-relationship model is a widely used model
that provides a convenient graphical representation to view data,
relationships and constraints \citep{silberchatz2006sistema}.
\begin{description}
    \item[Entity \newline] {An entity is a "thing" or "object" in the real world that is distinguishable from all other objects. An entity has a set of properties and the values for some set of properties may uniquely identify an entity. An entity may be concrete, such as a person or a book, or it may be abstract, such as loan, or a holiday, or a concept.}
    \item[Entity Set \newline] {An entity set is a set of entities of the same type that share the same properties, or attributes. The pictorial representation of the E-R model is the E-R diagram. E-R diagram can express the overall logical structure of a database graphically. Such a diagram consists of the following major components \citep{silberchatz2006sistema}:\begin{itemize}
        \item {{\bf Rectangle:} It represents entity sets.}
        \item {{\bf Ellipse:} It is used to represent attributes.}
        \item {{\bf Diamond:} It represents relationship sets.}
        \item {{\bf Line:} It links attributes to entity sets and entity sets to relationship sets.}
        \item {{\bf Double Ellipse:} It denotes multi-valued attributes. }
        \item {{\bf Dashed Ellipse:} It denotes derived attributes. }
        \item {{\bf Double Line:} It indicates total participation of an entity in a relationship set.}
        \item {{\bf Double Rectangle:} It represents weak entity sets.}
    \end{itemize}        }
\end{description}




\subsection{Schema Diagram}
In general, a relation schema consists of a list of attributes and
their corresponding domains. A schema diagram shows the graphical
representation of relation schema.

The schema diagram of our used relation schema is shown below.
\subsection{Data Flow Diagram}
The flow diagram in the next page depicts the overall design and
working steps of our designed  website.

\subsection{Use Case Diagram}
A use case is a set of scenarios that describing an interaction
between a user and a system.  A use case diagram displays the
relationship among actors and use cases.  The two main components
of a use case diagram are use cases and
actors\citep{mellor2004mda}.

\begin{figure}[htbp]
  \centering
 % \scalebox{0.5}{\includegraphics*[0.5in,8.5in][2.5in,10.5in]{./Figures/logo.eps}}
\includegraphics[height=3in]{./Figures/actor.eps}
   % \rule{35em}{0.5pt}
  \caption[Actor and User case]{Actor and User Case}
  \label{fig:Actor}
\end{figure}

An actor is represents a user or another system that will interact
with the system you are modeling.  A use case is an external view
of the system that represents some action the user might perform
in order to complete a task\citep{mellor2004mda}.
\subsubsection{Use of Use Case Diagram}
Use cases are used in almost every project.  The are helpful in
exposing requirements and planning the project. During the initial
stage of a project most use cases should be defined, but as the
project continues more might become visible\citep{mellor2004mda}.
\section{Features of the Proposed System}
\subsection{Student Basic information}
 Student basic information includes
student name, student's father name, mother name, local guardian
and resident or nonresident information, Department name,
permanent address and present address etc.  The higher authority
can create a new department as need. Student scholarship: Student

\subsection{scholarship information} Student scholarship information
is handled through the proposed system. Scholarship money withdraw
information are also kept in this system.

\subsection{Hall Fund management}
 The system provides the hall staff
to manage the student different kind of fund.  It is known that
different hall deals with different kind of fund type. So the
system has given a facility to create a new fund type.

\subsection{Student seat allocation and de allocation}
 The proposed
system effectively manages the seat allocation and de-allocation.
By using the system one can at first observe the status of a room.
The system keeps s information the status of every bed of a room.
The system also provides the facilities to create a room and the
number of beds.
\subsection{ Student card information} The card
issue to the student is one of the important functions the hall
authority. The proposed system will maintain the information about
different type of card issued by the hall.


\subsection{Teacher/Staff basic information}
 A hall management can
not be considered without keeping at least staff basic
information. The system will handle the staff basic information
such as name, address, designation and function.
\subsection{Notice board} The system will maintain a notice board. One can put
notice in a fixed format.

\subsection{Search} The proposed system provides the flexibility of
searching. The user does not need to write the search full key. It
will provide the facilities of the autosuggestion that is when one
write one or two character all entries matching will show as drop
down style.


\subsection{Some extra Features} Besides the above mentioned
feature the proposed system has also other feature such as follows

\begin{itemize}\item{About DU hall}
\item{ Admission information} \item{ Hall library information}

\item{Hall Recreation Achievement} \item{Form}
\end{itemize}

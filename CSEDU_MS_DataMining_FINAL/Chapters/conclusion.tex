\chapter{Conclusions}
\lhead{Chapter 5. \emph{Conclusions}}
\section{Research Summary}
In this thesis, we have proposed new window-batch based probabilistic strategy for finding frequent patterns from uncertain dynamic data. We proposed  \emph{U\textsuperscript{cap}} that is the upper bound of existential probability. We proposed new \emph{US-tree} data-structure that is very compact. For mining, we have proposed \emph{USFP-growth} algorithm that will efficiently and recursively mine frequent patterns from \emph{US-tree}.\\
For calculating \emph{U\textsuperscript{cap}} value we have taken upper bound of existential probability that makes the node sharing between same items. Fro uncertainty property of data node sharing was very much irregular in existing approaches. In our strategy, this sharing becomes more and that makes the tree more compact and efficient. Our proposed \emph{US-tree} is very compact for node possible node sharing much more than existing tree structures (e.g. \emph{SUF-growth} ~\cite{suf_growth}). To handle the stream of data we have proposed to divide the whole transactions into batches and windows that is completely dependent on the system user how much recent data she/he wants to store in the tree. Instead of keeping the support we have kept new meta information based on \emph{U\textsuperscript{cap}} value that helps further mining. We have developed new algorithm \emph{USFP-growth} mining algorithm that efficiently remove unnecessary patterns from the tree earlier that helps to gain running time efficiency. Conditional candidate tree while mining will be less than other approaches. This makes the mining algorithm very much faster. We have also introduced an efficient strategy to remove the false positives generated by our algorithm. Our comprehensive result study shows that total time (tree construction, tree mining, and false positive reduction) is less than existing approaches(e.g. \emph{SUF-growth} ~\cite{suf_growth}). We also developed a frequent tree pattern tree which may later be used to mine closed patterns and maximal patterns.
\section{Limitations and Scope of Future Studies}
As the frequent pattern mining over the domain of uncertain stream data is a very new, there are some scopes to extend and use our proposed approach as a tool for further research.\\ 
\textbf{First,} as we have introduced probabilistic model for calculating \emph{U\textsuperscript{cap}} value. This value can be studied and updated to find both upper and lower bound of existential probability.\\
\textbf{Secondly,} As we data-set is the stream which is dynamic it can change dynamically. Once an item comes to the top of the \emph{US-tree} stays there at least it becomes oldest data although may be very much infrequent in the recent data. This case may make the tree not compact that was possible. As data stream can not be read more than once. So no previously assumption can be done to this tree. So some re-construction after new batch inserted into the window will possibly be the one of the major optimization of the tree.
\section{Conclusions}
 % SMALL.TEX -- Released 5 July 1985
 % USE THIS FILE AS A MODEL FOR MAKING YOUR OWN LaTeX INPUT FILE. % EVERYTHING TO THE RIGHT OF A % IS A REMARK TO YOU AND IS IGNORED
 % BY LaTeX.
 %
 % WARNING! DO NOT TYPE ANY OF THE FOLLOWING 10 CHARACTERS EXCEPT AS
 % DIRECTED: & $ # % _ { } ^ ~ \

% \documentclass[12pt,a4paper,oneside]{report}% YOUR INPUT FILE MUST CONTAIN THESE

 %\begin{document} % TWO LINES PLUS THE \end COMMAND AT
 % THE END

\chapter{Conclusion}
  % THIS COMMAND MAKES A SECTION TITLE.
 \label{Chapter 7}
 \lhead{Chapter 7. \emph{Conclusion}}
 \section{Conclusion}
 From our daily life experience, it is seen that the staff of any organization in our country is not so sincere. From the organization necessary information can not be easily gathered as we think. The hall is such kind of organization which deals with huge amount of information about student. Here Student has to face the difficult to get their necessary information from the hall authority they make unnecessary delay to let the student know any kind of information more over the existing system forces them to become boring when they have to response a lot of request. All these matters promote us to build a web based Hall Management System. Information technology offers the mankind to store and retrieve the information in easiest way. Certainly a hall management system makes the overall working procedure of a hall more dynamic. We have tried our best to make the system more flexible and to offer the maximum feature of a hall. The most important feature of the designed system is that it is not built on specific hall. We studied a few halls and tried to shape the system to be usable for any hall.


 To implement the project, the methodologies of software engineering have been applied as much as possible. Although it has not been possible for us to follow the software engineering fully because of not the nature of problem not allowing us, yet we have implemented the system by following every step of water fall model.
Software maintenance is one of the most intimidating tasks that
require clear knowledge about the functionality of the software.
Those who are not used to using software or have not minimum
knowledge on software often show less eager to take the advantage
of the software. Moreover they are accustomed to using traditional
method of accounting information. Most of them fear to use
software by thinking of its maintenance cost high. Someone thinks
that to run software a man power enriched with IT knowledge is
required. All these maters create obstacle of applying software in
every sector.


By consider all these problems, our system has been started. We
tried to present a system that will shorten the burden of the task
of the hall staff. We have given the facilities of creating new
dimension of the work as the feel the necessity to add new
department or faculty. In future it s feature can be enhanced so
much easily.



Although we did not follow any specific normalize form but we have
tried to design the E-R diagram in such way so that no data
redundancy occur. It is known that a good E-R diagram design does
not demand any normalized form technique followed. When we
carefully define the E-R diagram, identifying all entries
correctly, the relation schemas generated from the E-R diagram
should not need much further normalization.  Besides this, for the
sake of performance, occasionally database designers choose a
schema that has redundant information; that is, it is not
normalized. They use the redundancy to improve performance for
specific applications. The penalty paid for not using a normalized
schema is the extra work (in terms of coding and execution time)
to keep the redundant data consistent.



The three tire architecture of a software has been followed to
implement the system. In three tire architecture, a software
solution should have three independent modules- the User Interface
(UI), Business Logic Layer (BLL) and the Data Access Layer (DAL).
The User interface, as its name suggests should provide the user
with interface to the data and different functionality of
software. The Business Logic Layer should handle all the necessary
logical calculations. Data Access Layer should handle all the data
related tasks. These layers will communicate with each other via
parameters but not interfere with each others task. This way a
software holds the provision of future modification and
scalability. HTML for the UI, PHP for the BLL and MySQL have been
used for the DAL.  But it may not fulfill all the requirements of
three tire architecture as a novice software engineer and as a
first work of our software engineering as a business class
software.



In fine, we would like to say that it is not possible to design a
system that fulfills its all requirements. We have tried to
implement a system that at least fulfilled the partial requirement
of a hall management system. The works has introduced us the shape
of the real world. To implement the design we have to face many
realities that must work as a benchmark for our future success.
The work has taught us how to collect information and to
compromise with time. It is no need saying that it will remain us
one step ahead in the field of career.




% \end{document} % THE INPUT FILE ENDS LIKE THIS

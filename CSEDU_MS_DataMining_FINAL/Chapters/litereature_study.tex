%\documentclass[a4paper,12pt]{book}
%\usepackage{fixltx2e}
%\usepackage{graphicx}	
%\begin{document}
\chapter{Background Study and Related Works}
\lhead{Chapter 2. \emph{Background Study and Related Works}}

In this chapter we have described some related work and provide some background stuffs that are  appropriate to the remainder of this paper. We have discussed about different terminology and related analysis for better understanding. We have focused the recent research works on uncertain and stream database. We have provided the analysis of existing algorithms, approaches their philosophy, working procedure, complexity analysis, aim and limitations that will help very much for better understanding and improvement of existing approaches.

\section{Data Mining}
In the previous chapter, Chapter 1 Introduction, we have described about data mining, frequent pattern mining, frequent pattern mining from uncertain data, frequent pattern mining from stream data. First one is candidate generation and test based approach named Apriori ~\cite{apriori}. Second one is pattern growth approach named as FP-growth ~\cite{fp_growth}. Apriori ~\cite{apriori} is a prior knowledge based  algorithm, that is , if any pattern is not frequent then one of its super-pattern must not be frequent.It works in a level wise approach. From the given database in each level it's generate frequent sub-patterns and merges them to propagate the candidates for next level. In this approach multiple times scan of database is mandatory and, a lot of test cases needed to be tested and for each check it is mandatory to search the database again and again. On the other hand FP-growth ~\cite{fp_growth} has been adopted divide and conquer approach. This has solved multiple scans and also reduced the huge candidate test. In this approach, firstly the huge database is compressed and put into a data structure named FP-tree holding the  itemset association information and divide such a compressed database into a set of condition databases, each associated with one frequent item, and mine the constructed conditional database recursively. Here, there is no need to generate candidate for next level or no need to generate candidate. Apriori suffers from huge candidate generation whereas FP-growth has resolved this problem. However, Apriori ~\cite{apriori} suffers much more when support factor. Thus all the approaches derived or influenced by Apriori ~\cite{apriori} (e.g U-priori ~\cite{u_priori}) also suffers form this problem.

\section{Uncertain Stream Data}
Data uncertainty is an inherent property in various applications due to reasons such as hardware fault, uncertainty of incidents, outdated sources or imprecise/incorrect measurement. In the age of Big data, uncertainty is one of the defining characteristics of data. Data is constantly growing in volume, variety, velocity and uncertainty. In a large range of applications domains uncertain data is found in abundance.  Today on the world wide web, data from sensor networks like GPS, weather ditection machines, within enterprises both in their structured and unstructured sources, biological data, meteorological trends, of medical behavior of living organisms data, human behavior, are the example of uncertain data. Customer name of a super shop is an real life example of uncertain data. For example there exists several customers named Mrs./Mr X registered in a super shop but if we want a particular Mrs./Mr. X then we will not certainly be able to find her/him by name.\\ \\
In modern age there are much more instruments we are being used to like smart-phones, smart-gears (e.g. smart watch, smart glass). All the devices has many more sensors those are collecting data each and every micro second. But all the data from the souces are imprecise/incorrect. For example a temperature sensor gives some information like current temperature is 40 degree, with \%5-10\% error. This makes this data imprecise and uncertain. Moreover, features like weather forecast is inherently uncertain. For example , todays weather forecast may be rainy but it has only 65\% confidence. There are many cases when information acquired from human experiences are not reliable and precise. Different people can describe same incident in different view and perspective. As an example, in a bird viewing event Mr. C saw a crow flying by. But, Mr. D thinks it was a raven. So, uncertainty values can be attached with these observations like the bird has 75\% probabilty to be crow and 20\% probability to be a raven.
%\documentclass{article} 
%\usepackage{graphicx}  
%\usepackage{multirow}
%\usepackage[table]{xcolor}
%\usepackage{fixltx2e}
%\usepackage{array}
%
%\begin{document}
\begin{table}[ht]
\centering

\begin{tabular}{|l|l|l|l|l|}
\hline
No & \multicolumn{4}{c|}{Items in Transaction} \\ \hline \hline
	T-1 & a(0.9) & c(0.6) & d(0.5) & e(0.2)\\\hline
	T-2 & a(0.9) & b(0.4) & e(0.1) & --    \\\hline
	T-3 & a(0.2) & c(0.9) & d(0.7) & --    \\\hline
	T-4 & b(0.3) & c(0.9) & -- & --\\\hline
	T-5 & a(0.1) & b(0.3) & c(0.9) & --    \\\hline
	T-6 & a(0.9) & e(0.3) & -- & --        \\\hline
   	T-7 & a(0.1) & d(0.6) & e(0.2) & --    \\\hline
	T-8 & a(0.1) & c(0.2) & f(0.6) & --    \\\hline
	T-9 & c(0.2) & d(0.9) & f(0.6) & --    \\\hline

	T-7 &  --  &  --  &  --  & --    \\\hline
	T-8 &  --  &  --  &  --  & --    \\\hline
	T-9 &  --  &  --  &  --  & --    \\\hline
\end{tabular}
\label{tab:ex_u}
\caption{Uncertain Stream Transaction Table}

\end{table}
%\end{document}
\subsection{Categorization of Uncertain Data}
By type of uncertainty assignment uncertain data can be categorized into three types. They are:
\paragraph{Attribute Uncertainty}
In attribute uncertainty, each uncertain attribute in a tuple has its own independent probability distribution. For example, if readings are taken of temperature and wind speed, each would be described by its own probability distribution for correctness and precision, as knowing the reading for one measurement would not provide any information about the other.
\paragraph{Correlated Uncertainty}
In correlated uncertainty, multiple attributes may be described by a joint probability distribution. For example, if there rain occurs then an musical festival event will be stopped. Then the occurrence of event depends on probability of rain. And thus this makes a co-relation with rain and musical festival
\paragraph{Tuple Uncertainty}
In tuple uncertainty, all the attributes of a tuple are subject to a joint probability distribution. This covers the case of correlated uncertainty, but also includes the case where there is a probability of a tuple not belonging in the relevant relation, which is indicated by all the probabilities not summing to one. For example, assume we have a tuple T\textsubscript{i} = \emph{(a, 0.4) | (b, 0.5)} and the probability this tuple exists in the data base is 10\%. Then this is called tuple uncertain.
		\begin{figure}
		\centering
			\includegraphics[width=1\textwidth]{../images/d_probability}
		\caption{Discrete Probability Distribution}
		\label{figure:d_probability}
		\end{figure}
		\begin{figure}
		\centering
			\includegraphics[width=1\textwidth]{../images/c_probability}
		\caption{Continuous Probability Distribution}
		\label{figure:c_probability}
		\end{figure}
\subsection{Uncertainty Types}
Uncertainty can be of two types: discrete and continuous. In this uncertainty  the probability of an uncertainty is discrete value (figure \ref{figure:d_probability}). Continuous uncertainty means the probability of an item/event being certain lies between a certain value (eg. 0 and 1) (figure \ref{figure:c_probability}). In many real-life applications, uncertain objects are already given by discrete observations, in particular if the objects are derived from sensor signals. This type of representation is motivated by the fact, in many cases, only discrete but ambiguous object information as usually returned by common sensor devices is available, e.g., discrete snapshots of continuously moving objects.

\subsection{Expected Support for Uncertain Data}
For uncertain data expected support calculation is not streight forward. The support calculation follows the following equation.
	\begin{equation}
	\emph ExpSup \qquad = \qquad \sum_{i = 0}^{UDB} [\prod_{x \in I } p(x , t_i)]
	\end{equation}
\emph {where,}	\textbf{\emph {I}} is itemset,	\textbf{\emph { p(x, ti)}} is existential probability value for any item \textbf{\emph {x}} in transaction \textbf{\emph {t\textsubscript{i}}} 	\textbf{\emph {UDB}} is an uncertain database.\\
For example let table \ref{table:uncertain_stream_transaction} be consider a uncertain database. Here expected support of itemset \emph{ab} will be $0+0.9*0.4+0+0+0.3*0.1+0+0+0+0=0.39$.

\subsection{Stream Data}
Stream data is data unbounded and continuous data incoming from any data sources that is producing all time a lots of data. These data is that much continuous and extreme in volume that storing this much data in any storage can not be thought. This much data processing and storing makes a huge cost that can be think. This stream of data is valueless unless this is processed, categorized and extract important knowledge. But this garbage can be diamond if this data is categorized and extract important information. Here data mining plays an important role.\\ \\
 For example day by day with enormous increase of internet availability and use the data producing each day by each user in every micro second is extremely huge. These big data computation is much more difficult. Particularly we can look at an example: as the social media is being popular day by day people are using these media like Facebook ~\cite{facebook}, Twitter \cite{twitter} etc. They are being used to these medias. Passing much more times and this gives a great opportunity to get a very much valuable information regarding their likes, their daily routine, their current status, their friends and families. This gives much more opportunity to profile people, categorized them. This gives the service selling organizations a great scope to digital online marketing, advertising their products and offer many promotional offers to people. Again, as this is used to communicate with each other then any terrorist activity can be predict earlier from these resources found from web. But the main difficulty is to extract information to categorized and profile these extreme data incoming from source.

\section{Existing Approaches}
Many algorithms has been developed to mine frequent itemsets from uncertain databases. Among them some are Apriori ~\cite{apriori} like approach and others are FP-growth ~\cite{fp_growth} like approach. U-Apriori ~\cite{u_priori} was proposed to mine uncertain frequent pattern that is Apriori like candidate generation and test based approach. UF-growth ~\cite{uf_growth}, UFP-growth ~\cite{ufp_growth} was proposed to mine frequent patterns from uncertain data but this suffers much more for compactness of tree. CUF-growth ~\cite{cuf_growth}, CUF-growth\textsuperscript{*} ~\cite{cuf_growth}, PUF-growth ~\cite{cuf_growth} was proposed later for mining uncertain data but they produce false positives as they works with probabilistic uncertainty modeling. All of them are unable to mine stream of uncertain data. UF-stream ~\cite{suf_growth} and SUF-growth ~\cite{suf_growth} proposed to mine uncertain frequent pattern from stream. UF-stream ~\cite{suf_growth} suffers from both false positives and false negatives. The SUF-growth ~\cite{suf_growth} is an exact mining algorithm that has resolved the false positive and false negative genration but the constructed tree structure is like UF-growth ~\cite{uf_growth} that suffers much because the data structure (tree ) SUF-growth ~\cite{suf_growth} use to keep information is not compact. Before elaborate discussion of SUF-growth ~\cite{suf_growth} we will provide a brief discussion on background of uncertain stream mining approaches.

	
	\subsection{APriori}
	Apriori ~\cite{apriori} is one of the common algorithm for data mining. It works in a level wise approach based on a prior knowledge. In each level Apriori algorithm generates frequent sub-patterns from the given database and merges them to generate the candidates for next level. That's why this kinds of algorithms called candidates generating algorithm.  Apriori is designed to operate on databases containing transactions (for example, collections of items bought by customers, or details of a website frequentation). Generating frequent sub-patterns it can use large itemset property. The sub-patterns are easily parallelized and it's easy to implement. For those reasons from it's proposed time Apriori is of the frequent used algorithm for data mining.	But, some of it's draw back many other algorithms took place over it. The draw back are given below.In APriori, number of database scan increases with the dimension of the candidate itemset.It needs transaction databases in memory resident for assumed performance. Performance suffers much in candidate set generation and test. This approach fit for frequent pattern finding from certain data but not for uncertain or stream data.
	
	\subsection{FP growth }
	FP-growth ~\cite{fp_growth} is a tree based approach that keeps meta data information in a compressed tree structure. It named the tree structure is FP-tree. In this tree node share is possible. So the data structure becomes very compressed. It can store much more data of the transaction database. Later mining algorithm is used to extract frequet patterns. This has solved the candidate generation of Apriori ~\cite{apriori} algorithm and also the huge test. Fp-growth mining algorithm use divide and conquer approach for candidate frequent sub-pattern mining. It first generates the candidate sub-pattern tree and then mine recursively. This is one of the very first idea of frequent pattern mining. This approach also designed for certain data and does not fit for data stream.
	
	\subsection{U-Priori}
	U-Apriori ~\cite{u_priori} is a  modification of Apriori ~\cite{apriori} algorithm which is proposed to handle uncertain data. We already know in Apriori ~\cite{apriori} algorithm support count play an important role. Number of databases scan and candidate for next level depends on support count. The modification of Apriori in U-Apriori is support count. It has designed for handling support using minimum support calculation of uncertain data. Specially, instead of incrementing the support counts of candidate patterns by their actual support, U-Apriori increments the support countsof candidate patterns by their given support count using the uncertain support count equation. Though U-Apriori come over Apriori to solve support count problem of uncertain data but U-Apriori suffers from the some problems. As U-Apriori is a modification of the Apriori algorithm, performance of U-Apriori algorithm for large scale of data because it follows level wise sub-patterns generation and merge to generate candidate for next level. That's why number of databases scan increase with the dimension of itemset. If the existential probabilities of most items within a pattern I are small, increments for each transaction can be insignificantly small. Consequently, many candidates would not be recognized as infrequent until most transaction were processed.  
	
	\subsection{UF-growth}
	Observing outperforms of FP-growth ~\cite{fp_growth} over Apriori ~\cite{apriori}, UF-growth ~\cite{uf_growth} was proposed for mining uncertain data. We know key to success of FP-growth over Apriori is FP-tree, which is a compact tree structure capturing frequent items within transactions in the databases of precise data. Like FP-growth, UF-growth also used the tree construction approach. This constructs UF-tree By extracting appropriate tree paths to construct subsequent FP-trees, frequent itemset can be mined. Each tree path represents a transaction. Each node in a tree path captures(i) an item x and (ii) its actual support(i.e, occurrence count of x in that tree path). Tree paths (from the root) are merged if they share the same items(i.e., the captured tow items share a node if they has both same item id and same existential probability). Due to this path sharing, the UF-tree is usually compact. However, as dealing with uncertain data, the situation is different form certain data. The expected support of any itemset X is the sum of products of existential probability of items within X. Hence, UF-growth uses a UF-tree to capture frequent items within transactions of uncertain data. Each node in an UF-tree captures (i) an item x, (ii) its existential probability value, and (iii) the occurrence count of x in that tree path. By doing so, UF-growth finds all and only those frequent itemsets by computing the expected support of an itemset X (as the sum of products of the captured existential probability values). Tree paths are merged if they share the same items and existential probability values. Consequently, UF-trees may not be as compact as FP-trees. In real life scenirio this tree is never compact. And mining suffers much more as the candidate for sub-pattern tree grows as the transaction grows.
	
	\subsection{UFP-growth}
	To reduce the tree size in UF-tree of UF-growth ~\cite{uf_growth}, UFP-growth ~\cite{ufp_growth} was proposed. In this approach, items are grouped and share same node (i.e., nodes with the same x but similar existential probability values) into a cluster. Each cluster of the item x captures (i) the maximum existential probability value of all nodes within the cluster and (ii) the number of existential probability values in each cluster. Depending on the clustering parameter, the resulting tree namely, UFP-tree may be as large as the UF-tree (i.e., no reduction in tree size). On the other hand, if the UFP-tree  is smaller than the UF-tree, then UFP-growth may return approximate results (e.g., with false positives or infrequent itemsets). Though it solves some tree compactness problem but the tree is still not that much compact. More over this do not fit for uncertain stream data. 
	
	\subsection{PUF-growth}
	To reduce the size of the UF-tree ~\cite{uf_growth} and UFP-tree ~\cite{ufp_growth}, the prefix-capped uncertain frequent pattern tree, PUF-tree structure and mining algorithm PUF-growth ~\cite{puf_growth} was proposed, in which important information about uncertain data is captured so that frequent patterns can be mined from the tree. The PUF-tree  is constructed by considering an upper bound of existential probability value for each item  which is named as I\textsuperscript{cap}. When generating a k-itemset (where $k>1$). This upper bound of an item x\textsubscript{r} in a transaction t\textsubscript{j} is called the (prefixed) item cap, I\textsuperscript{cap} of x\textsubscript{r} in t\textsubscript{j}. Thus PUF-tree is very much compact. This approach generates some false positive. But the false positive reduction technique makes this a stronger approch. But this is not fit for uncertain stream data. 
	
	\subsection{UF-streaming}
	UF-streaming ~\cite{suf_growth} was proposed to mine frequent item from uncertain stream data. This is a window and batch based approach that capture most recent data, as most recent data is most important. It has divided the transaction streamd data in several batch each batch is inserted into tree structure that is UF-tree ~\cite{uf_growth} and mine each tree using UF-growth ~\cite{uf_growth} and put all the frequent patterns found in a UF-stream structure. For avoiding false negative they have used a value preMinSup less then mininum support (0 < preMinSup < minimum support). Then untill window is completed the frequent itemset is put into UF-stream tree structure. This has addressed the problems of stream uncertain data mining but it suffers from some problems given below.
	\begin{itemize}
		\item Un-necessary mining each batch. Let window size is 3 and batch size is 2. If one wants the mining result at window 50 then in this approach it is needed to mine each of 1 to 50 batch for getting the result.
		\item The used tree structure is UF-tree that suffers from compactness problem.
		\item Extra data structure for keeping frequent itemset UF-stream tree structure.
		\item False positive generation in the mining process.
		\item False negative generation in the mining process.
		\item This has high running time and memory consumption.
	\end{itemize}
	
	\subsection{SUF-growth}
	SUF-growth ~\cite{suf_growth} addressed the limitations exist in UF-streaming ~\cite{suf_growth}. SUF-growth algorithm outperforms over UF-streaming algorithm. It improves over UF-streaming by avoiding the aforementioned potential problems for mining frequent itemset from streams of uncertain data. SUF-growth is an exact algorithm. Its means SUF-growth returns only truly frequent itemset. Where UF-streaming returns both true or false frequent itemset. Its use only minsup. No need to use preMinsup. There also have any problem of finding an appropriate value of preMinsup. This algorithm does not need UF-stream structure to store the mined itemsets. Need transaction databases in memory resident for assumed performance. Instead of this its use delayed mode for mining. As a result unnecessary computation could be reduced and unnecessary mining is avoided.
		\begin{figure}[]
		\centering
			\includegraphics[width=1\textwidth]{../images/suf_simulation}
		\caption{SUF-growth ~\cite{suf_growth} tree construction}
		\label{figure:suf_simulation}
		\end{figure}
	Considering the advantages of SUF-growth over others algorithm there is question. The is how SUF-growth ~\cite{suf_growth} algorithm find frequent itemsets from streams of uncertain data using a new tree structure called SUF-tree. Now we describe about this. We first construct a SUF-tree, and then extract relevant paths from this SUF-tree (which is a global tree) to recursively form smaller UF-trees for projected databases. Due to the dynamic nature and Property 2 of data streams, expected support of items is continuously affected by the arrival of new batches (and the removal of the contents of older batches). Arranging items in frequency-dependent order in the SUF-tree may lead to swapping—which, in turn, can cause merging and splitting—of tree nodes when the global frequencies of items change. Hence, in the SUF-tree, items are arranged according to some canonical order (e.g., lexicographic order), which can be specified by the user prior to the construction of the SUF-tree or the mining process. Consequently, the SUF-tree can be constructed using only one scan of the streams of uncertain data, and the resulting SUF-tree captures the contents of the streams. Moreover, the SUF-tree preserves the usual tree properties. The occurrence count of a node is at least as high as the sum of occurrence counts of its children.The ordering of items is unaffected by the continuous changes in the expected support values of items.\\ \\
	For example for table \ref{table:uncertain_stream_transaction} transaction database let construct the SUF-tree. Figure \ref{figure:suf_simulation} shows the construction of the SUF-tree. Here as the suff tree is based on UF-tree construction approach the node sharing is very rare. From figure \ref{figure:suf_simulation} its clearly visiable that the node sharing is not that much impressive as the node should be shared if the two same item has same existential probability and in real life scenirio this sharing is very rare. For this reason \emph{a(0.9)} in \emph{T\textsubscript{1}} and \emph{a(0.2)} in \emph{T\textsubscript{3}} did not share single node and the tree compactness could not be accomplished.
	Although the SUF-growth has resolved some limitations of UF-streaming ~\cite{suf_growth}, some limitations still exists. The findings are listed below:
	\begin{itemize}
		\item The tree structure it uses is UF-tree ~\cite{uf_growth} structure which suffers from compactness. So SUF-growth tree also inherit this limitation.
		\item The tree construction cost will be much more (both running time and memory) as the transaction grows.
		\item The mining algorithm it uses is FP-groth like approach that generates a huge candidate sub-pattern tree that costs much in the mining time. It increase mining cost (both running time and memory).
 	\end{itemize}
	
\section{Summary}
The approaches for finding frequent patterns on uncertain data is challanging because of data uncertainty property. Althogh many of the approaches has been adopted to address and solve this constrains but still there are limitations. Moreover there is still limitations in mining uncertain stream. The compressed data structure designe is a tough challange for uncertain stream data. Later chapters we will provide a novel, completely new and effecient data structure that will make the tree compactness more and gain both running time and memory efficiency.
%\end{document}
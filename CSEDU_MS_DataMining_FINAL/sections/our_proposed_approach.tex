\documentclass[a4paper,12pt]{book}
\usepackage{fixltx2e}	
\usepackage[english]{babel}	
\usepackage[nottoc]{tocbibind}
\usepackage{graphicx}  
\usepackage{multirow}
\usepackage[table]{xcolor}
\usepackage{fixltx2e}
\usepackage{caption}
\usepackage{subcaption}
\usepackage{array}


\begin{document}
\chapter{Our Proposed Approaches}
\input{pa_intro.tex}
\newpage


\section{Analysis Uncertain Stream Data Properties}
Hello World
%\documentclass{article} 
%\usepackage{graphicx}  
%\usepackage{multirow}
%\usepackage[table]{xcolor}
%\usepackage{fixltx2e}
%\usepackage{array}
%
%\begin{document}
\begin{table}[ht]
\centering

\begin{tabular}{|l|l|l|l|l|}
\hline
No & \multicolumn{4}{c|}{Items in Transaction} \\ \hline \hline
	T-1 & a(0.9) & c(0.6) & d(0.5) & e(0.2)\\\hline
	T-2 & a(0.9) & b(0.4) & e(0.1) & --    \\\hline
	T-3 & a(0.2) & c(0.9) & d(0.7) & --    \\\hline
	T-4 & b(0.3) & c(0.9) & -- & --\\\hline
	T-5 & a(0.1) & b(0.3) & c(0.9) & --    \\\hline
	T-6 & a(0.9) & e(0.3) & -- & --        \\\hline
   	T-7 & a(0.1) & d(0.6) & e(0.2) & --    \\\hline
	T-8 & a(0.1) & c(0.2) & f(0.6) & --    \\\hline
	T-9 & c(0.2) & d(0.9) & f(0.6) & --    \\\hline

	T-7 &  --  &  --  &  --  & --    \\\hline
	T-8 &  --  &  --  &  --  & --    \\\hline
	T-9 &  --  &  --  &  --  & --    \\\hline
\end{tabular}
\label{tab:ex_u}
\caption{Uncertain Stream Transaction Table}

\end{table}
%\end{document}
\subsection{Stream Property}
\subsection{Uncertainty Property}
For expected support calculation in an uncertain database we get the following equation\\
%\documentclass{article}
%\usepackage{fixltx2e}
%\begin{document}
\begin{equation}
\emph ExpSup \qquad = \qquad \sum_{i = 0}^{UDB} [\prod_{x \in I } p(x , t_i)]
\end{equation}
\begin{center}


\textbf{\emph {where,}}\\ 
\begin{itemize}
\item
\textbf{\emph {I}} is itemset,
\item
\textbf{\emph { p(x, ti)}} is existential probability value for any item \textbf{\emph {x}} in transaction \textbf{\emph {t\textsubscript{i}}} 
\item
\textbf{\emph {UDB}} is an uncertain database.

\end{itemize}
\end{center}
%
%\end{document}
\newpage
\section{Preliminaries}
\paragraph*
This is Definitions and Preliminaries.
\input{../example/table_uncertain_transaction_batch.tex}
\newpage
\section{Mining Frequent Patterns from Uncertain Databases}
Our proposed algorithm is divided into four parts. (1) Giving each item in a transaction a prefix value is called \emph{U\textsuperscript{cap}}. (2) insert transaction into \emph {US-tree}. (3) sliding the \emph {US-tree} (4) mining the \emph {US-tree} (\emph{USFP-growth}) and (5) Eliminating false positive (not frequent but exists in frequent item set) . For simulating our approach we consider Table~\ref{tab:ex_u} as uncertain stream data. For this simulation we consider window size as 2 and batch size 3. That means 3 transactions creates a batch and 2 batches create a window. After completing window construction (inserting batch 1 and 2 the tree is completed. When new transactions comes we slide the window. That means we remove oldest batch batch 1 and put batch 2 as the old batch. Then insert new batch in the tree as batch 3. So for window size 2 the tree always contains 2 batches. Thus the tree always holds the newest information. In next subsections we will elaborately explain our approach of every steps.
\subsection{Assigning Prefix Value}
%\documentclass{article}
%\usepackage{fixltx2e}
%\begin{document}
\subsection*{}
\noindent
In this section we will describe the batch and window grouping and the prefix value \emph{U\textsuperscript{cap}} calculation. First we calculate batch and window calculation and grouping our transaction data. As we describe earlier for the example Table-\ref{table:uncertain_stream_transaction} easy simulation we take batch size as \emph{$3$} and window size as \emph{$2$}. So first \emph{$3$} transactions \emph{T\textsubscript{1}}, \emph{T\textsubscript{2}} and \emph{T\textsubscript{3}} are grouped and labeled as \emph{Batch-1}. Then next \emph{$3$} \emph{T\textsubscript{4}}, \emph{T\textsubscript{5}} and \emph{T\textsubscript{6}} are grouped together and labeled as \emph{Batch-2}. Next \emph{$3$} \emph{T\textsubscript{7}}, \emph{T\textsubscript{8}} and \emph{T\textsubscript{9}} are grouped together and labeled as \emph{Batch-3}. Thus the consecutive next \emph{$3$} transaction should be grouped as batch and ready to be inserted into \emph{US-tree}. As our window size is $2$, after inserting two batches into the \emph{US-tree} the window will be completed. Before inserting next batch \emph{Batch-3} we need to remove \emph{Batch-1}(oldest one) from tree and move \emph{Batch-2} to \emph{Batch-1 's} position and then insert new batch, \emph{Batch-3}. Thus the latest information is inserted and kept into the \emph{US-tree}. Table-\ref{table:transaction_batch} shows the window and batch grouped for stream transaction example Table-\ref{table:uncertain_stream_transaction}.

\subsection*{}
\noindent
For this calculation we earlier proposed an equation \ref{eqn_ucap}. From this equation we can create \emph{U\textsuperscript{cap}} for each item in a transaction. And thus for all transactions in the data stream. To calculate one transaction, for each item in a transaction, if item is the first item in a transaction than item's existential probability is its \emph{U\textsuperscript{cap}} value, otherwise item's \emph{U\textsuperscript{cap}} is max of previous items existential probability multiplied by item's  own existential probability. For example let Table-\ref{table:transaction_batch} T1 is \emph{a(0.9), c(0.6), d(0.5), e(0.2)}. 
In this transaction item \emph{a(0.9)} is the first item. So its \emph{U\textsuperscript{cap}} is 0.9. 
For second item, \emph{c(0.60)} previous item is only \emph{a(0.90)}. So c's $\emph{U\textsuperscript{cap}} = 0.9*0.6 = 0.54$. 
For third item, \emph{d(0.50)} there are two items before it, \emph{a(0.9)} and \emph{c(0.6)}. Among them \emph{a} has max existential probability, that is \emph{0.9}. So \emph{d's } $\emph{U\textsuperscript{cap}} = 0.9*0.5 = 0.45$. 
For fourth item \emph{e(0.2)} there are three items before it, \emph{a(0.9)} ,\emph{c(0.6)} and \emph{d(0.5)}. Among them \emph{a} has max existential probability is \emph{0.9}. So \emph{e's}  $\emph{U\textsuperscript{cap}} = 0.9*0.2 = 0.45$. Thus we can calculate each item's \emph{U\textsuperscript{cap}} value for a transaction. For easier understanding we have calculated all the item's \emph{U\textsuperscript{cap}} of Table-\ref{table:transaction_batch} and put into Table-\ref{table:prefix_assigned}.
%\end{document}
\subsection{US-tree Construction}
\paragraph*{}
\emph{U\textsuperscript{cap}}  

In this section we will describe about our tree construction \emph{US-tree}. We have describe earlier about our batch and window. A batch should be inserted into the tree in it's own window slot. After inserting all batches of to full the window, the window shall be complete and be ready to mine. We said earlier section that our tree will be very compact. For this we have proposed an approach for sharing nodes. For sharing same tree nodes two items with same id and order should not care about own existential probability. If item is already in the tree with same id then two items should share the node. Thus the tree will be very compact.When inserting an item in the tree the \emph{U\textsuperscript{cap}}  of the tree should be updated by adding the prefix value of the node. Thus each batch should be inserted into the tree. FOR EXAMPLE TABLE REF. We 
\subsection{Sliding Window}
\documentclass{article}
\begin{document}
\paragraph*{}
In this section we will describe elaborately about our mining procedure. An important fact in our data type is transaction comes in as a stream and that's why we have created a window based 
\end{document}


\subsection{Mining US-tree : FPUS-growth}
%\documentclass{article}
%\usepackage{caption}
%\usepackage{graphicx}
%\begin{document}
\subsection*{}
In this section we will discuss about our mining algorithm \emph{USFP-growth}(algorithm-\ref{algorithm:mine}) That will find frequent patterns. For mining we used \emph{FP-growth} like approach. Generally, we can remove all the nodes having support less than \emph{minimum support}. From headr table we get such information that which nodes are less than \emph{minimum support}. We told earlier that for \emph{U\textsuperscript{cap}} value we took the upper limit, we also can we can remove all the nodes having prefix value less than \emph{minimum support}. In this process we can eliminate most of the infrequent nodes from tree. For this purpose we used header table that was created when creating the \emph{US-tree}. Then we construct conditional tree starting from the lowest support holding node. From header table we also get the position of all nodes containing same item in the tree. 
\subsection*{}
Lets mine the tree we constructed earlier. Figure-\ref{figure:min_before} is \emph{US-Tree} for mining and corresponding header table before starting mining. From the header table we get that support of \emph{a} is $3.00$, \emph{b} is $1.00$, \emph{c} is $3.30$, \emph{d} is $1.20$, \emph{e} is $0.60$. So \emph{e} is not frequent for one item set frequent pattern. So it is sure that no item set contains \emph{e} will be frequent. That is the basic upward closure property of frequent item set. So we remove all the nodes of \emph{e} and get the new mining tree Figure-\ref{figure:min_ready}. And its corresponding header. Here we find all one item sets that are frequent and that is \emph{\{a\}, \{b\}, \{c\} \{d\}} Now we will construct conditional tree for the found frequent one item and mine the conditional tree. But items having total \emph{U\textsuperscript{cap}} less than \emph{minimum support} is not needed to construct conditional tree because this \emph{U\textsuperscript{cap}} value has been take as the upper bound. So total \emph{U\textsuperscript{cap}} value less than \emph{minimum support} indicates that item must not be exists in the $2$ or more frequent item set. So we do not construct conditional tree for these items.
\subsection*{}
We construct conditional tree from items having lowest total \emph{U\textsuperscript{cap}} value greater than \emph{minimum support}. As \emph{b} having total \emph{U\textsuperscript{cap}} is $.69$ \emph{b} is ignored for constructing conditional tree. Then the next candidate is \emph{d}. From header table pointer we find that there is only one path item \emph{d} exists in the mining tree Figure-\ref{figure:min_ready}. That is \{\emph{a, c, d}\}:$1.08$. So we create conditional tree and update all nodes mining probability with \emph{d's} \emph{U\textsuperscript{cap}} $1.08$. For this conditional tree Figure-\ref{figure:d_cond} the header tables says all the nodes in the tree are having \emph{U\textsuperscript{cap}} greater than \emph{minimum support} $.9$. So all the items are ready to be constructed as conditional tree. As here is only one branch so we do not further construct conditional tree and take all the combinations as frequent items Figure-\ref{figure:d_cond} Table. so we find \emph{\{dc\}, \{da\}, \{dca\}} as frequent pattern. Next we create conditional tree for \emph{c}. Here c exists in the tree for three path those are \emph{\{a, c\} : $0.72$ , \{a , b, c\} : $.027$ and \{b, c\} : $0.27$}. So we create conditional tree (Figure-\ref{figure:c_cond}). Total mining value of \emph{c} is the sum of item caps in each respective path of \emph{c}. From the header we see that  \emph{b} has total cap having less than \emph{minimum support} so we remove \emph{b} and create two item set \emph{\{ca\}}. Next we construct \emph{ca} conditional tree that contains only root (\emph{\{\}}). So no further tree is needed to be constructed and minned. Than we create conditional tree for \emph{a}. And find only root \emph{\{\}}. Thus we get all the  frequent patterns. All the patterns are \emph{\{a\}, \{b\}, \{c\} \{d\}, \{dc\}, \{da\}, \{dca\} and \{ca\}}. As we have found all patterns from upper bound, so that it is guaranteed that there will be no false negative. But some false positive can be exists in the found frequent item set as there may be some value less than max value we assumed. so in the next section we will show an approach that eliminates the all false positives (if exists) in the data set.
%\documentclass{article}
%\usepackage{caption}
%\usepackage{graphicx}
%\begin{document}
\begin{figure}
\begin{minipage}{0.40\textwidth}
  \centering
  
	\begin{center}
	\begin{tabular}{ |c|c|c| } 
 	\hline
 		Item&\emph{U\textsuperscript{cap}}&Support\\ \hline\hline
 		A &  3.00  & 3.00\\ \hline
 		C &  1.26  & 3.30\\ \hline
 		D &  1.08  & 1.20\\ \hline
 		B &  0.69  & 1.00\\ \hline
\end{tabular}
\end{center}  
  
  
  \captionof{table}{Header Table for }
\end{minipage}
\hfill
\begin{minipage}{0.40\textwidth}
  \centering
  \includegraphics[width=\textwidth]{../images/M_TREE.jpg}
  \captionof{figure}{\emph{US-tree} Before Mining}
\end{minipage}


\caption{\emph{US-tree} and its Header Table}
\end{figure}
\begin{figure}

\begin{minipage}{0.40\textwidth}
  \centering
  
	\begin{center}
	\begin{tabular}{ |c|c|c| } 
 	\hline
 		Item&\emph{U\textsuperscript{cap}}&Support\\ \hline\hline
 		A &  3.00  & 3.00	\\ \hline
 		C &  1.26  & 3.30	\\ \hline
 		D &  1.08  & 1.20	\\ \hline
 		E &  0.54  & 0.60	\\ \hline
 		B &  0.69  & 1.00	\\ \hline
\end{tabular}
\end{center}  
  
  
  \captionof{table}{Header Table }
\end{minipage}
\hfill
\begin{minipage}{0.40\textwidth}
  \centering
  \includegraphics[width=\textwidth]{../images/sim_06.jpg}
  \captionof{figure}{\emph{US-tree} for mining.}
\end{minipage}
\caption{\emph{US-tree(Mining)} and its Header Table}
\end{figure}
%\end{document}
\documentclass{article}
\usepackage{caption}
\usepackage{graphicx}
\begin{document}
\fbox{ 
{\centering
\begin{minipage}{0.40\textwidth}
  \centering
  
	\begin{center}
	\begin{tabular}{ |c|c| } 
 	\hline
 		Item&Value\\ \hline\hline
 		a &  1.08  	\\ \hline
 		c &  1.08   	\\ \hline
 		
\end{tabular}
\end{center}  

  
  \captionof{table}{\emph{d-cond tree} Header }
\end{minipage}
  \hfill
\hfill
\begin{minipage}{0.23\textwidth}
  \centering
  \hfill
  \includegraphics[width=.8\textwidth, height=5cm]{../images/D_COND.jpg}
  \captionof{figure}{\emph{d-cond} Tree}
  \hfill
  
\end{minipage}
\hfill
\begin{minipage}{0.30\textwidth}
  \centering
  
	\begin{center}
	\begin{tabular}{ |c| } 
 	\hline
 		Freq Patterns \\ \hline\hline
 		dc  	\\ \hline
 		da   	\\ \hline
 		dca   	\\ \hline
 		
\end{tabular}
\end{center}  

  
  \captionof{table}{ \emph{Frequent Patterns for d} }
\end{minipage}
}
}

\end{document}

%\end{document}
\subsection{False positive reduction}


\paragraph{This Is Elemininition Paragraph}
%\documentclass{article}
%\usepackage{fixltx2e}
%\usepackage{caption}
%\usepackage{graphicx}
%\begin{document}
\begin{figure}
\begin{minipage}{0.60\textwidth}
  \centering
  
	\begin{center}
	\begin{tabular}{ |c| } 
 	\hline
 		Frequent Items\\ \hline\hline
 		a \\ \hline
 		b \\ \hline
 		c \\ \hline
 		d \\ \hline
 		dc \\ \hline
 		da \\ \hline
 		dca \\ \hline
 		ca \\ \hline
\end{tabular}
\end{center}  
  
  
  \captionof{table}{\emph{Frequent Items}\\(with false positive)}
\end{minipage}
\hfill
\begin{minipage}{0.40\textwidth}
  \centering
  \includegraphics[width=\textwidth]{../images/frequent_tree.jpg}
  \captionof{figure}{\emph{Frequent Item Tree} }
\end{minipage}
\end{figure}
\begin{figure}
\begin{minipage}{.6\textwidth}
  \centering
  
	\begin{center}
	\begin{tabular}{ |c|c| } 
 	\hline
 		No & Items \\ \hline\hline
 		a  & \emph{a(0.9),c(0.6),d(0.5)}\\ \hline
 		b & \emph{a(0.9),b(0.4),e(0.1)}\\ \hline
 		c & \emph{a(0.2),c(0.9),d(0.7)}\\ \hline
 		d & \emph{b(0.3),c(0.9)}\\ \hline
 		dc& \emph{a(0.1),b(0.3),c(0.9)} \\ \hline
 		da & \emph{a(0.9),e(0.3)
}\\ \hline
\end{tabular}
\end{center}  
  
  
  \captionof{table}{Transaction Table after removing Infrequent Items}
\end{minipage}
\hfill
\begin{minipage}{0.40\textwidth}
  \centering
  \includegraphics[width=.8\textwidth]{../images/frequent_tree_final.jpg}
  \captionof{figure}{\emph{Frequent Item Tree} identifying false positives}
\end{minipage}

\end{figure}
%\end{document}


\subsection{Algorithm}
\documentclass[a4paper]{article}

\usepackage[english]{babel}
\usepackage[utf8]{inputenc}
\usepackage{amsmath}
\usepackage{amsfonts}
\usepackage{graphicx}
\usepackage[colorinlistoftodos]{todonotes}
\usepackage{algorithm}
\usepackage{algpseudocode}
\usepackage{geometry}

\begin{document}
\paragraph*{}
Our proposed algorithm is given here.
%\documentclass{article}
%\usepackage[]{algorithm2e}
%\begin{document}
\noindent
 \textbf{\emph{Input :}} \emph{Transaction T}\\
 \textbf{\emph{Output : }}\emph{Transaction T with U\textsuperscript{cap} assigned} \\
\begin{algorithm}
 \textbf{\emph{Procedure : }}\emph{calculateU\textsuperscript{cap}}\\
 \For{$I=0;I<T.size;I++$}{
 	\If{$I==0$}{
 		$I.$\emph{I\textsuperscript{cap}}$I.Probability$\\
 		$max = I.Probability$
 	}
 	\Else{
		$I.$\emph{I\textsuperscript{cap}}$I.Probability*max$\\
		\If{$I.Probability>max$}{
			$max = I.Probability$
		}
 	}
 }
 \caption{\emph{U\textsuperscript{cap} Assignment Algorithm} }
 \label{algorithm:cap_assignment}
\end{algorithm}
%\end{document}
%\documentclass{article}
%\usepackage[]{algorithm2e}
%\begin{document}
\noindent
\textbf{\emph{Input :}} \emph{Window Size w, Batch Size b, Uncertain Transactions UDB}\\
 \textbf{\emph{Output : }}\emph{Root of Constructed US-tree} \\
\begin{algorithm}[H]
 
 \textbf{\emph{Procedure : }}\emph{constructUS-tree}\\
 Create Root node \{\} \\
 \For{$i\leftarrow w$}{			 
	\For{$j\leftarrow b$}{
		$T \leftarrow read next Transaction$ \\
		$T \leftarrow calculate $\emph{U\textsuperscript{cap}}\\
		$parent \leftarrow root$  \\
		\For{$I \leftarrow T.item$}{
			\If{$I.id$ \textbf{in} $parent.childes$}{
				$node \leftarrow parent.child$ \\
				$node[w].prefix=node[w].prefix+I.$U\textsuperscript{cap}
				
			}\Else{
				$node\leftarrow$ \emph{new node()} \\
				$node[w].prefix=I.$U\textsuperscript{cap}\\
				$parent.addChild(node)$\\
				$parent \leftarrow node$
			}
			\emph{Update Header Table}\\
			
		}
	 }
 }
 \emph{mine USFP-growth() }\\
 \emph{slide tree}\\
 $counter = 0$ \\
 \While{$T \leftarrow has more Transaction$}{
 	\If{$counter==b$}{
 		\emph{mine USFP-growth()}\\
 		\emph{slide tree(root)}\\
 		$counter=0$
 	}
 	\If{$I.id$ \textbf{in} $parent.childes$}{
				$node \leftarrow parent.child$ \\
				$node[w].prefix=node[w].prefix+I.$U\textsuperscript{cap}
				
	}\Else{
		$node\leftarrow$ \emph{new node()} \\
		$node[w].prefix=I.$U\textsuperscript{cap}\\
		$parent.addChild(node)$\\
		$parent \leftarrow node$
	}
	\emph{Update Header Table}\\
 	
 }
 
 \caption{\emph{US-tree} construction Algorithm}
 \label{algorithm:tree_construction}
\end{algorithm}
%\end{document}
%\documentclass{article}
%\usepackage[]{algorithm2e}
%\begin{document}
\noindent
 \textbf{\emph{Input :}} \emph{root of US-tree, min\underline{ }sup}\\
 \textbf{\emph{Output : }}\emph{Frequent Pattern tree} \\
\begin{algorithm}
 \textbf{\emph{Procedure : }}\emph{mineUS-tree}\\
 $FrequentRoot = new Node$ \\
 $H\leftarrow$\emph{US-tree.Header}\\
 \For{$I\leftarrow H$}{
 	\If{$I<$\emph{min\underline{ }sup}}{
 		\For{$node\leftarrow I.Item$}{
 			$parent = node.parent$\\
 			$parent.removeChild(node)$
 		}
 	}\Else{
 		$FrequentRoot.add(I.id)$
 	}
 } 	

 \emph{SortByTotalPrefix(H)}\\
 \For{$I\leftarrow H>$\emph{min\underline{ }sup}}{
 	$cRoot= $\emph{CreateConditionalTree($I$)}
	\emph{mine(cRoot,I,min\underline{ }sup)} 	
 }
 \textbf{\emph{Procedure : }}\emph{mine(condTreeRoot cRoot,Item I,min\underline{ }sup)}\\
 \While{$cRoot!=null$}{
 	\emph{removeInfrequentFromHeader()}\\
 	\emph{Update $FrequentRoot$}\\
 	$H\leftarrow condTreeRoot.Header$\\
 	\For{$I_2 \leftarrow H$}{
 		$I=I+I_2$
 		\emph{mine(cRoot,I,min\underline{ }sup)}
 	}
 }
 \caption{\emph{USFP-growth Algorithm}}
 \label{algorithm:mine}
\end{algorithm}
%\end{document}
\end{document}
\section{Summary}
\input{pa_summary}

\end{document}
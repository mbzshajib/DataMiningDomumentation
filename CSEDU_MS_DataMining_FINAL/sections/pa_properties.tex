\subsection{Stream Property}
Stream data are such data those come and go away. They can not be stored in any patterns. Data streams are continuous and unbounded. This property makes finding patterns so much difficult. As when stream flows away we lose them, we have no choice to store data and scan more than one time we have to find some technique to store valuable information that will help to find valuable information later. For example table-\ref{table:uncertain_stream_transaction} shows the stream transaction database. Here \emph{T\textsubscript{1}, T\textsubscript{2}, T\textsubscript{3}, T\textsubscript{4}, T\textsubscript{5}, T\textsubscript{6}, T\textsubscript{7}, T\textsubscript{8}, T\textsubscript{9}} comes one after another and goes away. It may occur that when \emph{T\textsubscript{10}} comes after a long time of \emph{T\textsubscript{4} or T\textsubscript{5} or T\textsubscript{6}} came. So there is no way get \emph{T\textsubscript{4}} when \emph{T\textsubscript{10}} comes. As we are always interested in most recent data, because most recent data are most valuable, we proposed a sliding window based approach that holds the most recent data in our proposed \emph{US-tree} to hold valuable information for further valuable information extraction.
\subsection{Uncertainty Property}
Data is not always precise. Hardware limitations, loss of information during transmission, sampling errors etc can make precise data uncertain. For this reason the data existence is not for sure. Each time data can comes with a probability value that is called its existential probability. For example in table-\ref{table:uncertain_stream_transaction}  we can see each transaction having items with item's probability. This value is its existential probability. Let for transaction-\emph{T\textsubscript{2}} there is four items \emph{a(0.9), c(0.6), d(0.5)} and \emph{e(0.2)}. Here \emph{0.9, 0.6, 0.5, 0.2} all are existential probability of corresponding \emph{a, c, d, e}. That means \emph{a's} existential probability is \emph{0.9}. Probability of existence of \emph{a} in that transaction is \emph{0.9}. This property makes mining very much difficult. This property says that in different transaction in a transaction data base same item can have different existential probability. So finding similarity between same item becomes another matter to worry about. For expected support calculation of item becomes very much tough. For expected support calculation equation-\ref{equation:exp_sup} is used.
%\documentclass{article}
%\usepackage{fixltx2e}
%\begin{document}
\begin{equation}\label{equation:exp_sup}
\emph ExpSup \qquad = \qquad \sum_{i = 0}^{UDB} [\prod_{x \in I } p(x , t_i)]
\end{equation}
\begin{center}


\textbf{\emph {where,}}\\ 
\begin{itemize}
\item
\textbf{\emph {I}} is itemset,
\item
\textbf{\emph { p(x, ti)}} is existential probability value for any item \textbf{\emph {x}} in transaction \textbf{\emph {t\textsubscript{i}}} 
\item
\textbf{\emph {UDB}} is an uncertain database.

\end{itemize}
\end{center}
%
%\end{document}
From this equation we can see that calculation of support is not that easy and straight forward like certain data. Certain data support is calculated just counting the total existence of that items in the transaction were as for uncertain data it depends on its existential probability. It may occur that some data set exists many times in a transaction database but with a low probability, so ultimately the data must not be frequent. For example for table-\ref{table:uncertain_stream_transaction} if we want to find the support of \emph{ae} then we need to do the following calculation. For \emph{T\textsubscript{2}} $0.9*0.2=0.18$, \emph{T\textsubscript{2}} $0.9*0.1=0.09$, \emph{T\textsubscript{3}} $0.0$, \emph{T\textsubscript{4}} $0.0$, \emph{T\textsubscript{5}} $0.0$, \emph{T\textsubscript{6}} $0.9*0.3=0.27$, \emph{T\textsubscript{7}} $0.1*0.2=0.02$, \emph{T\textsubscript{8}}=$0.0$ and \emph{T\textsubscript{9}}=$0.0$.\\$Sup_{ae}\\=Sup_{ae(T_1)}+Sup_{ae(T_2)}+Sup_{ae(T_3)}+Sup_{ae(T_4)}+Sup_{ae(T_5)}+Sup_{ae(T_6)}+Sup_{ae(T_7)}+Sup_{ae(T_8)}+Sup_{ae(T_9)}\\=0.18+0.09+0.0+0.0+0.0+0.27+0.0+0.02+0.0+0.0=0.54$\\ So if \emph{minimum support is $0.9$} than \emph{ae} is not frequent but \emph{ae} exists four times in the transaction. So it makes very much difficult to find which is important pattern and which is not.
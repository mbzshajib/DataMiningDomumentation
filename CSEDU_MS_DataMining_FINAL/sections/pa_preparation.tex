%\documentclass{article}
%\usepackage{fixltx2e}
%\begin{document}
\subsection*{}
\noindent
In this section we will describe the batch and window grouping and the prefix value \emph{U\textsuperscript{cap}} calculation. First we calculate batch and window calculation and grouping our transaction data. As we describe earlier for the example Table-\ref{table:uncertain_stream_transaction} easy simulation we take batch size as \emph{$3$} and window size as \emph{$2$}. So first \emph{$3$} transactions \emph{T\textsubscript{1}}, \emph{T\textsubscript{2}} and \emph{T\textsubscript{3}} are grouped and labeled as \emph{Batch-1}. Then next \emph{$3$} \emph{T\textsubscript{4}}, \emph{T\textsubscript{5}} and \emph{T\textsubscript{6}} are grouped together and labeled as \emph{Batch-2}. Next \emph{$3$} \emph{T\textsubscript{7}}, \emph{T\textsubscript{8}} and \emph{T\textsubscript{9}} are grouped together and labeled as \emph{Batch-3}. Thus the consecutive next \emph{$3$} transaction should be grouped as batch and ready to be inserted into \emph{US-tree}. As our window size is $2$, after inserting two batches into the \emph{US-tree} the window will be completed. Before inserting next batch \emph{Batch-3} we need to remove \emph{Batch-1}(oldest one) from tree and move \emph{Batch-2} to \emph{Batch-1 's} position and then insert new batch, \emph{Batch-3}. Thus the latest information is inserted and kept into the \emph{US-tree}. Table-\ref{table:transaction_batch} shows the window and batch grouped for stream transaction example Table-\ref{table:uncertain_stream_transaction}.

\subsection*{}
\noindent
For this calculation we earlier proposed an equation \ref{eqn_ucap}. From this equation we can create \emph{U\textsuperscript{cap}} for each item in a transaction. And thus for all transactions in the data stream. To calculate one transaction, for each item in a transaction, if item is the first item in a transaction than item's existential probability is its \emph{U\textsuperscript{cap}} value, otherwise item's \emph{U\textsuperscript{cap}} is max of previous items existential probability multiplied by item's  own existential probability. For example let Table-\ref{table:transaction_batch} T1 is \emph{a(0.9), c(0.6), d(0.5), e(0.2)}. 
In this transaction item \emph{a(0.9)} is the first item. So its \emph{U\textsuperscript{cap}} is 0.9. 
For second item, \emph{c(0.60)} previous item is only \emph{a(0.90)}. So c's $\emph{U\textsuperscript{cap}} = 0.9*0.6 = 0.54$. 
For third item, \emph{d(0.50)} there are two items before it, \emph{a(0.9)} and \emph{c(0.6)}. Among them \emph{a} has max existential probability, that is \emph{0.9}. So \emph{d's } $\emph{U\textsuperscript{cap}} = 0.9*0.5 = 0.45$. 
For fourth item \emph{e(0.2)} there are three items before it, \emph{a(0.9)} ,\emph{c(0.6)} and \emph{d(0.5)}. Among them \emph{a} has max existential probability is \emph{0.9}. So \emph{e's}  $\emph{U\textsuperscript{cap}} = 0.9*0.2 = 0.45$. Thus we can calculate each item's \emph{U\textsuperscript{cap}} value for a transaction. For easier understanding we have calculated all the item's \emph{U\textsuperscript{cap}} of Table-\ref{table:transaction_batch} and put into Table-\ref{table:prefix_assigned}.
%\end{document}